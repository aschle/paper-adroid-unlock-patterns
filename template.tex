\documentclass[twocolumn, a4paper, 10pt]{article}
\usepackage[cmex10]{amsmath}

% \usepackage{helvet}
% \renewcommand{\familydefault}{\sfdefault}
% \fontfamily{phv}\selectfont

\usepackage[T1]{fontenc}
\renewcommand*\familydefault{\sfdefault}

% makes everything a bit tighter
\usepackage{microtype}

\usepackage{amsopn}
\usepackage{amsthm}
\usepackage{amsmath}

\usepackage{url}
\usepackage{hyperref}

\usepackage{graphicx}
\graphicspath{{./figures/}}
\DeclareGraphicsExtensions{.pdf,.jpeg,.png,.eps, .svg}

% if you want to draw sth: have a look at tikz
\usepackage{tikz}
\usetikzlibrary{positioning}
\usetikzlibrary{calc, fit, shapes, decorations.markings, calendar}	

\begin{document}
\title{
	Improving the Security of Android Unlock Patterns by giving Feedback to the User during the Password Creation Process}

\author{
	Alexa Schlegel
}

\maketitle


\def\abstractname{{\textbf Abstract}}
\begin{abstract}
{
\bfseries
An Android Unlock Pattern is a graphical password scheme, widely adopted for unlocking the screen on Android smartphones. Instead of using a PIN or a textual password, the user can set up an unlock pattern by connecting dots in a $3\times3$ grid.

In theory, an Android Unlock Pattern is more secure than a five-digit PIN. According to several studies, users tend to pick easy-to-guess patterns so the security of user-chosen patterns is close to a three-digit PIN scheme. To overcome the problem of weak passwords, textual password schemes have integrated password composition policies. This, in general, strengthens security but sometimes can negatively affect usability when HCI\footnote{Human Computer Interaction} principles are disregarded.

The following research proposal aims to introduce password creation policies into graphical passwords with no negative impact on usability. The user will receive  constructive feedback, in form of hint, concerning the creation of a stronger pattern during password creation. A user study will be conducted.
}
\end{abstract}


\section{Introduction}
\label{sec:intro}

Authentication on smartphones needs to be done on a regular basis for unlocking the device. Different manufacturers implemented various authentication methods. Widely used and well known are textual passwords and PIN numbers. Taking into account the distribution of smartphones today, Android phones are dominating the market with about 78.0\%\footnote{\url{http://www.idc.com/prodserv/smartphone-os-market-share.jsp}, 31.08.2015 - 12:52PM} of the market share. The Android Unlock Pattern, a \textit{recall-based} graphical password scheme, is the default authentication choice and therefore used very often. According to recent studies, about 50\% of Android users are using Android Unlock Pattern~\cite{VanBruggen:2013:MSU:2501604.2501614, van2014studying}.

Like textual passwords, a graphical password scheme is a \textit{knowledge-based} authentication mechanism in which users enter a shared secret as evidence of their identity. During enrollment, the user has to choose a pattern with four to nine dots and during the authentication phase, needs to recall the pattern and draw the path on the screen. The user can select a path according to the following rules:

\begin{enumerate}
	\small
	\setlength\itemsep{0em}
	\item four to nine dots have to be used
	\item one dot can be used only once
	\item dots are connected with a straight line
	\item one cannot jump over dots not visited before
\end{enumerate}

Textual passwords are typically difficult to remember, also depending on length, and are predictable if users are allowed to choose passwords. Graphical password schemes have been proposed as a alternative to overcome those usability and security issues. ``The reduced memory burden will facilitate the selection and use of more secure or less predictable passwords, but it is now clear that the graphical nature of schemes does not, by itself, avoid the problems typical of text password systems.''~\cite{Biddle:2012:GPL:2333112.2333114}

The \textit{theoretical password space}~(TPS) is the total number of unique passwords that can be generated according to the given rules; in contrast the \textit{effective password space}~(EPS)  is the number of passwords in the TPS that are likely to be chosen by real-world users.~\cite{forget2007persuasion} The TPS of Android Unlock Pattern contains 389,112 possible patterns~\cite{Aviv:2010:SAS:1925004.1925009}, which makes it in theory more secure than a 5-digit PIN scheme.

Uellenbeck et al.~\cite{Uellenbeck:2013:QSG:2508859.2516700} were the first to demonstrate the skewed distribution of Android Unlock Patterns, e.g. the bias in starting point and $n$-grams, that make user chosen patterns guessable. They showed that 50\% of the patterns could be guessed with only 1,000 guesses: this corresponds to an EPS of a 3-digit PIN scheme for half of the Android users. Building mainly on Uellenbeck's study, research has been done related to the Android Unlock Pattern focusing on various topics:~\cite{Sun2014308, siadati2015fortifying, Aviv:2014:UVP:2664243.2664253, Andriotis:2013:PSS:2462096.2462098}. Those papers will be discussed later in detail.

In contrast to textual passwords, which made are available via password leaks, patterns are only collected from in-lab studies (using devices and/or pen\&paper) or from web-based studies (self-reporting or web applications).\footnote{Aviv et al. ``Comparisons of Data Collection Methods for Android Graphical Pattern Unlock'' poster at SOUPS 2015} Some studies~\cite{siadati2015fortifying, Aviv:2014:UVP:2664243.2664253} acquired participants from from Amazon Mechanical Turk~(MTurk) or did user tests with university students~\cite{Uellenbeck:2013:QSG:2508859.2516700, Sun2014308}.

The security of Android Unlock Patterns can be improved either by (a) increasing the TPS or (b) expanding the EPS. Different methods have been applied and evaluated, including password strength meters\footnote{Indicating password strength during creation process using visual or textual representations for week, medium and strong.}~\cite{Sun2014308, siadati2015fortifying}, a random starting point~\cite{siadati2015fortifying}, alternative patterns (e.g. circle)~\cite{Uellenbeck:2013:QSG:2508859.2516700} and increased grid size\footnote{Aviv et al. ``Do bigger grid sizes mean better passwords? 3x3 vs. 4x4 Grid Sizes for Android Unlock Patterns'', poster at SOUPS 2015}.

To the best of my knowledge, no user feedback during the creation process has been evaluated or tested yet. Password strength meters offer real time feedback about the underlying security measurement, which is based on the mathematical model calculating a strength score, but it provides no advice for the user about what to change in order to accomplish a more secure pattern.

Password composition policies have been studied for textual passwords (e.g.,~\cite{Inglesant:2010:TCU:1753326.1753384, Komanduri:2011:PPM:1978942.1979321}). Applying policies usually results in stronger passwords. The policies can also cause bad usability and strange user behavior if they are applied too much. Real-time feedback has an positive impact on usability and can help users create strong passwords with fewer errors~\cite{Shay:2015:SSI:2702123.2702586}.

I want to transfer password composition policies to graphical passwords in order to increase security, while not decreasing usability. This would be a contribution towards more secure mobile devices.


\section{Research Questions}
\label{sec:question}
The purpose of this research is to find out whether password creation policies applied to graphical passwords (e.g. Android Unlock Patterns) lead to stronger user-chosen passwords, or rather extend the effective password space, with no negative impact on usability. The research questions can be formulated as follows:

\begin{description}
	\item[Q1]  Are user chosen patterns, created using a password composition policy, stronger (more secure) than patterns created in the conventional way?
	\item[Q2] Can password composition policies be applied to graphical passwords with similar implications for security and usability?
\end{description}
  
  
\section{Related Work}
\label{sec:related}
The following section gives an overview about graphical passwords in general, about possible security attacks, and discusses relevant studies about Android Unlock Patterns and its limitations. In addition the latest research about password composition policies and its impact on security and usability is summarized.


\subsection{Graphical Passwords}
\label{sec:related:grafical}

The first graphical password scheme was introduced by Blonder~\cite{blonder1996graphical} in 1996, followed by Draw A Secret (DAS) by Jermyn et al.~\cite{Jermyn:1999:DAG:1251421.1251422} in 1999. DAS was improved by using background images to make users create more complex passwords, called BDAS~\cite{Dunphy:2010:CLR:1837110.1837114}. Tao and Adams in 2008 invented Pass-Go~\cite{tao2008pass}, which is very similar to Android Unlock Patterns. For an extensive overview of graphical passwords, see the work of Biddle et al.~\cite{Biddle:2012:GPL:2333112.2333114} or Oorschot and Thorpe~\cite{Oorschot:2008:PMU:1284680.1284685} or for a  short summary, Sun et al.~\cite{Sun2014308}.

Oorschot and Thorpe~\cite{Oorschot:2008:PMU:1284680.1284685} improved DAS by recommending password rules based on password complexity properties:

\begin{quote}
	\small
	\begin{enumerate}
		\setlength\itemsep{0em}
		\item Require a stroke count of at least [\dots]
		\item Disallow passwords having global reflective (mirror) symmetry [\dots]
		\item Require at least one stroke of length 1.
	\end{enumerate}
\end{quote}


\subsection{Studies about Android Unlock Pattern}
\label{sec:relatedstudies}

The results and methods of recent studies are summarized and their limitations are explained. Table \ref{tab} gives an overview about the number of participant and their background.

\begin{table}[b]
	\small
	\centering

	\begin{tabular}{llccc}
		{\bf Year} & {\bf Author} & {\bf \#Participants} & {\bf Background} \\
		2013       & Uellenbeck       & 105/113/366    & Uni   \\
		2013       & Andriotis       & 144    & -  \\
		2014       & Sun     & 81  & Uni    \\
		2014       & Aviv       & 384  & MTurk     \\
		2015       & Siadati   & 270  & MTurk             
	\end{tabular}
			\caption{Overview about studies.}
			\label{tab}
	%
	% Beim ersten kompilieren musste ich diese Zeile merkwürdigerweise immer auskommentieren, danach lief es 
	%

\end{table}

\paragraph{2013--Uellenbeck et al. ~\cite{Uellenbeck:2013:QSG:2508859.2516700}}
conducted the first study about security of Android Unlock Patterns. Several studies on a university campus with in total 584 participants generating 2,900 patterns was carried out. It included a pen\&paper study with 105 participants to collect data about users' ``real world'' patterns. They observed a bias in starting point towards corners with 75\% probability and a tendency that people tend to choose adjacent points. They also found often-used typical sub-patterns.

Pattern strength is measured using \textit{partial guessing entropy}~\cite{bonneau2012science}, which measures the average number of guesses that the optimal attack needs in order to find a correct password (or just fraction of accounts). They trained a Markov-chain model for cracking passwords. They found out that the entropy is in between a 2-digit PIN scheme and a 3-digit PIN scheme. Based on those findings alternative patters (e.g. circle, random) were evaluated in a second user study with 366 participants.

Here, one drawback is that the underlying security model allows no conclusion about how to create a stronger password. Only the starting point problem is addressed here and $n$-grams, which would lead to dictionary checks. As the focus lies on security improvement, no usability aspects are considered at all.


\paragraph{2013--Andriotis et al.~\cite{Andriotis:2013:PSS:2462096.2462098}} 
This paper results in a mixed attack combining physical (trace of fingers on screen, replicating Aviv et al.~\cite{Aviv:2010:SAS:1925004.1925009}) and psychological (heuristics about how user set unlock patterns) attacks. A user study with 144 participants was conducted. Users had to create what they consider to be a ``secure'' and ``easy to remember'' pattern. The following parameters were analyzed for creating heuristics: average pattern length, number of direction changes, start and end points, sub-patterns with length one to four. The evaluation was based on Shannon's entropy. The secure pattern was longer and included more direction changes. In the end, the mixed attack was tested with 22 participants, resulting in the cracking of 20 of 22 patterns.


\paragraph{2014--Sun et al.~\cite{Sun2014308}}
started with a detailed statistical analysis of all possible pattern, looking at characteristics (number of dots, physical length, intersections, overlaps) and its distributions. Two different pattern strength meters were evaluated during a user study conducted on a university campus with 81 participants. They showed that a password strength meter, which gives instant feedback to the user during the creation process, had an positive effect on security. People using the password strength meter created passwords with more dots, longer lengths and more intersections. They state that pattern strength is largely determined by its visual complexity. For calculating entropy, Burr's formula~\cite{burr2004electronic} was modified based the  characteristics.

Although they analyzed characteristics, they do not transform the findings into constructive feedback about how to create a strong pattern. Also a distribution analysis of user chosen passwords with respect to characteristics is missing. The study lacks the evaluation of the memorability (usability) of the created passwords.

\paragraph{2014--Aviv et al.~\cite{Aviv:2014:UVP:2664243.2664253}}
This study focuses on the visual perception of usability and security. 384 participants (from MTurk) had to choose between two passwords (pairwise preference): the one that looked (a) more secure and (b) more usable. They found out that usability and security are inversely related. Visual features that can be attributed to complexity indicated a stronger perception of security, spatial features (shifts up/down, left/right) are not so strong indicators for security or usability. They built an logistic model to predict perception preference by focusing on features used in the survey and other related work. They achieved 70\%  of predicted preference. The strongest feature, they found out, is password length (sum of all euclidean length). They measured the following features: number of points, crosses (and exes), non-adjacent, knight-moves, height, side.

Perceived security is a good indicator, but is not identical to standard metrics for password strength. A conclusion could also be that users need to be educated about security. Also, perceived usability needs to be evaluated as to whether it applies in practice (memorability, error rates, and so on).

\paragraph{2015--Siadati et al.~\cite{siadati2015fortifying}}
increased the strength of user-chosen patterns by using a \textit{persuasive security framework}~\cite{forget2007persuasion, forget2008persuasion}, a set of principles to get users to behave more securely. They conducted a user study with 270 participants form MTurk to evaluate and test
two improvements expanding EPS: (1)~BLINK (suggested starting point) and (2)~EPSM (continues visual feedback during creation). By creating 60\% strong passwords with BLINK and 77\% strong passwords with EPSM, they raised security noticeably. They used the same Markov-chain model as Uellenbeck et al.~\cite{Uellenbeck:2013:QSG:2508859.2516700}

Here, too, no feedback is given about how to create a stronger password.
They state that user are already aware which patterns are more secure and therefore need no hits. In a small pre-study they found out that user perceived security is not the same as security in realty. This is definitely a conflict that needs to be further investigated and is in line with findings from Aviv et al.\cite{Aviv:2014:UVP:2664243.2664253}.


\subsection{Attacks on Graphical Passwords}
\label{sec:related:attacks}

Knowledge factor attacks on graphical passwords can be divided into:

\begin{description}
	\item[(1) guessing or psychological attacks] e.g. bias in patterns like skewed distribution in order to limit search space, dictionary attacks like textual passwords, brute force guessing
	\item[(2) capture or physical attacks] also called side-channel attacks, e.g., smudge attack, shoulder surfing
\end{description}

Smudge attacks are studied by Aviv et al.~\cite{Aviv:2010:SAS:1925004.1925009} in detail. Using photographs, they showed that it was possible to fully or partialy recover patterns. Generating a more complex pattern makes it automatically more resistant to (1) and (2), but this would need to be investigated further. 


\subsection{Password Composition Policies}
\label{sec:related:policies}
In the context of textual password,s a study by Shay et al. shows that real-time feedback during password creation helps the user to create stronger passwords with fewer errors.However, password policies may cause usability problems.~\cite{Shay:2015:SSI:2702123.2702586}. The Usability and security of passphrases\footnote{Passphrases are longer passwords consisting of multiple words.} is studied by Keith et al.~\cite{Keith200717}, who state that ``passphrases lead to more typographical errors, are perceived as more difficult to use, but are actually no more difficult to remember than other password methods.'' ``Password policies requiring length lead to more usability, and in some cases more security, than those requiring only a comprehensive mix of character classes and a dictionary check.''~\cite{Shay:2014:LPS:2556288.2557377}. Inglesant and Sasse~\cite{Inglesant:2010:TCU:1753326.1753384} conclude that ``rather than focusing password policies on maximizing password strength and enforcing frequency alone, policies should be designed using HCI principles to help the user to set an appropriately strong password in a specific context of use.'' Komanduri et al.~\cite{Komanduri:2011:PPM:1978942.1979321} found out that commonly held beliefs about password composition and strength are inaccurate:
\begin{quote}
	\small
	\begin{enumerate}
		\setlength\itemsep{0em}
		\item Adding numbers to passwords is thought to add little entropy; we found, by contrast, a lot of entropy in numbers.
		\item Dictionary checks, although other- wise useful, add much less entropy than expected.
		\item Unexpectedly, users typically create passwords that exceed minimum requirements, thus increasing password entropy
	\end{enumerate}
\end{quote}

Password composition policies do have an positive effect on security and, keeping HCI principles in mind, do not affect usability in a negative way.


\section{Security Measurement}
\label{sec:security}
The \emph{security} or \emph{strength} of a graphical password describes how hard it is for an attacker to guess or crack the pattern~\cite{Keith200717}. What people think or perceive as secure is not in line with what really is secure~\cite{Aviv:2014:UVP:2664243.2664253}. Different approaches to measure pattern strengths can be found in recent work:

\begin{description}
	\item[Guessing Entropy] Can be used to measure strength of password distribution. This measures the average number of guesses that the optimal attack needs in order to find the correct password.~\cite{massey1994guessing}
	
	\item[Partial Guessing Entropy ($\alpha$-guesswork)] by Bonneau~\cite{Bonneau:2012:QRP:2310656.2310722}. Finds the minimal number so that the guesses cover at least a fraction $\alpha$ of the passwords (used in~\cite{Uellenbeck:2013:QSG:2508859.2516700}).
	
	\item[Shannon's entropy] monograms, start and end points, entropy is calculated based on probability of point $X$ being selected in the pattern or being at start (or end), for $n$-grams conditional entropy is calculated (used in~\cite{Aviv:2014:UVP:2664243.2664253}).
	
	\item[Modified entropy formula] Burr's~\cite{burr2004electronic} entropy formula is modified for graphical password adding a score for visual complexity,  based on pattern characteristics (used in~\cite{Sun2014308}).
	
	\item[Score Function MM-score] score function $f(X)$ based on the probabilistic of a given pattern $X$, using the Markov model by Uellenbeck~\cite{Uellenbeck:2013:QSG:2508859.2516700}, so a more likely pattern gets a lower score, and a less likely one get a higher score (used in~\cite{siadati2015fortifying}).
\end{description}


\section{Pattern Characteristic}
\label{sec:characteristics}

Pattern characteristics or visual complexity features were looked at when doing pattern analysis. These characteristics are very promising in relation to giving user feedback. The following summarizes the features used in recent literature about Android Unlock Patterns and graphical passwords in general:

\begin{itemize}
	\setlength\itemsep{0em}
	\item start point
	\item end point
	\item size (number of connected dots)
	\item length (sum over all euclidian distances between dots)
	\item intersections (two groups: X crossings with angle 90 and others)
	\item overlaps (no crossing but touching)
	\item non-adjacent
	\item knight moves
	\item height
	\item side
	\item sub-patterns with different number of dots
	\item direction changes
	\item symmetry (vertical/horizontal)
\end{itemize}

In 1957, Attneave~\cite{attneave1957physical} studied the perceived complexity problem of shapes, and concluded that the complexity is related to the number of turns in the contour of the shape, the symmetry of the shape, and the variability of angular change between successive turns.


\section{Usability Measurement}
\label{sec:usability}

The term \emph{usability} describes how easy a password is for a user to both remember and correctly enter into a login prompt~\cite{Keith200717}. Usually usability and security are seen as counterparts, but the goal should be to increase usability and security simultaneously. For extensive recommendations regarding methods for evaluation of usability (e.g., login success rates, login times, password creation times) and security see~\cite{Biddle:2012:GPL:2333112.2333114}. Recommendations are given for lab, field and web-based studies. The most important elements of usability testing related to passwords are short and long term memorability tests.

\section {Methods to Improve Security}
\label{sec:improve}
This part summarizes the different methods to improve security found in literature. In general, there are two different approaches: (1) expand effective password space, (2) increase theoretical password space, in order to indirectly expand the effective password space. 

\begin{itemize}
	\setlength\itemsep{0em}
	\item manipulate pattern (e.g., remove top left dot)
	\item rearrange pattern (e.g., random -- no three points lie on one line, circle pattern)
	\item blacklisting (e.g., forbidding frequently used or easy to guess patterns, similar to dictionary checks at textual passwords)
	\item random assignment (e.g., random assigned pattern, random starting point)
	\item user education (e.g., password strength meter)
	\item change pattern size (e.g. from $3\times3$ to $4\times4$)
\end{itemize}

\section{Research Idea}
\label{sec:idea}
The aim is to improve the security of Android Unlock pattern while keeping usability in mind. Improving security can be achieved in different ways (see section~\ref{sec:improve}), but providing feedback to the user during the password creation process has not been evaluated yet. The password strength meter evaluated by Sun~\cite{Sun2014308} and Siadati~\cite{siadati2015fortifying} is a good starting point, but should be studied further. The underlying characteristics of patterns (see section~\ref{sec:characteristics}) can be used to derive constructive feedback, which will help the user to understand what to do to generate a stronger password. Research has shown that feedback has an positive effect on the security of a textual password (see section~\ref{sec:related:policies}), but this has to be evaluated for graphical passwords as well.

Furthermore, I think users are not completely aware of the rules on how to create a pattern. In particular, I do not think that all users know about crossings and overlying lines. This would need to be investigated as well.


\section{Research Plan}
\label{sec:plan}

My research plan consists of the following steps:

\begin{enumerate}
	\setlength\itemsep{0em}
	\item Conduct a pen \& paper study on pattern construction rules
	\item Evaluate summarized security measurements (see section~\ref{sec:security})
	\item Invent a new measurement or use existing security measurements
	\item Rate existing password characteristics~ (see section~\ref{sec:characteristics}) based on some criteria (which needs to be defined first)
	\item Derive different policies $P_1, P_2, \dots P_n$ from most promising characteristics
	\item Create prototype for feedback interface
	\item Test invented policies and feedback interface with pen \& paper
	\item Improve policies and feedback interface
	\item Conduct a user study\footnote{with following recommendations by Biddle et al.~\cite{Biddle:2012:GPL:2333112.2333114}}
\end{enumerate}

The most important and at the same time challenging part is to ``find'' a measurement for evaluating the influence of invented policies. The Markov Chain model using partial guessing entropy by Uellenbeck~\cite{Uellenbeck:2013:QSG:2508859.2516700} seems promising and will be used as a first attempt.

{
	\bibliographystyle{plain}
	\bibliography{tmp}
}

\appendix
\end{document}