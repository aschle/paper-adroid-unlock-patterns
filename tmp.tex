AUTHENTICATION FACTORS
Authentication is the act of confirming the correctness of an identity. There are three different categories of authentication, called factors. The (1) knowledge factors is something the user knows (e.g. password, PIN), (2) ownership factors is something the user has (e.g. ID card, security token like smart card) and (3) the inherence factor is something the user is or does (e.g. biometric data like fingerprint).[TODO/cite]



The following research plan consist of the following steps:\\

\begin{enumerate}
	\item Finding a useful metric
	\item Summarize all findings of recent studies related to Android Unlock Patterns
	\item Derive graphical password policies (P1 and P2)
	\item Test policies with pen \& paper (User Study)
	\item Summarize all findings related to feedback/guidance related to password policies (also find stuff related to mobile phones)
	\item Designing and implementing feedback and guidance to Android Unlock Pattern application
	\item Conducting a User Study
\end{enumerate}

First of all a metric for evaluating the strength of a patterns needs to be found. Looking at Uellenbeck and (maybe better) Sun, so I can just use one or create a better one. Based on the metrics, ideas for a good policy can be derived. To be more precise and accurate the chosen metric should be evaluated first!\\

Also based on the other findings in recent usability studies regarding Android Unlock Patterns (characteristics of patterns used in the real world, distribution of patterns in the real world) further ideas for policies can be derived.\\

The effectiveness and usefulness of the collected ideas for policies needs to be found out and rated by some criteria. The two different policies P1 and P2 need to be put together, based on the previous rating (if needed, not sure yet).\\

To get a first impression of the effect of policies a pen \& paper study should be conducted.\\

Based on the (maybe changed) policies, the guiding and feedback process can be designed and implemented keeping HCI principles in mind. [TODO: find some stuff on visual feedback/guidance to users on mobile devices, I think the study was just conducted for desktop users :(, but maybe principles can be transferred to mobile]\\

A user study should be carried out to evaluate the impact of policies in general and compare P1 and P2. Also the study should cover the memorability of created patterns (in contrast to the study conducted by Sun et. al.)\\